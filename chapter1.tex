%!TEX root = thesis.tex
\chapter{Introduction}
\label{intro}
% \SVN $Revision: 838 $
% \SVN $Author: sgordon $
% \SVN $Date: 2014-05-12 13:30:42 +0700 (Mon, 12 May 2014) $
% \SVN $URL: https://sandilands.info/svn/Common/Styles/siitthesis/chapter1.tex $

Opportunistic networks (OppNets) are one of the most interesting evolutions of multi-hop wireless network especially in Mobile Ad-hoc Networks (MANETs).
%%
On the one hand, MANETs characterize an approach to conceal the mobility of the nodes by constructing \emph{stable} end-to-end paths for communications. 
%%
On the other hand, opportunistic network consider a problem of node mobility in MANETS as an opportunity to exploit \cite{Conti2014}.
%% 
In this network scheme, mobile nodes are enabled to communicate with each other even without connected route and prior network topology knowledge \cite{Pelusi2006}.
%%
Several concepts behind opportunistic network come from the studies on DTN that led to the specification of its architecture \cite{Conan2008,Yu2012,Rongxing2010,Schurgot2012}. 
%%
Source and destination nodes might never be fully connected at the same time in opportunistic network, so the forwarding algorithms in such networks follow a \emph{store-carry-forward} paradigm \cite{Yamamura2011,Jie2007,Jie2007a} by exploiting opportunistically connections arising from mobility nature of nodes and temporary wireless links. 
%%
Typical algorithms differ based on their decisions as how to forwards the data, at what time the data is forwarded and to whom the data is sent \cite{Joe2010}. 
%%
However, the decision algorithms of what the data to sent has never fully implemented. 
%%
Messages are en route between the sender and the destination on the routes that dynamically built, and any possible node can opportunistically be used as next hop, provided it is likely to bring the message closer to the final destination.
%%
With the opportunistic paradigm, a data can be delivered from a source toward a destination by exploiting the sequence of connectivity graphs generated by the mobility of the nodes \cite{Acer20111,Ferretti2013}.

In the real world scenario, there are numerous examples of such networks implementing for specific applications. 
%%
These applications are mainly based on the effect from environments causing the extreme networks scheme.
%% talk about military
The interferences and jamming in the military operations are the example of opportunistic environments, thus there are many opportunistic routing proposals for military domain \cite{Kerdsri2012a,Scott2005,Kidston2012,Haillot2009}.
%% talk about animal
The chaotic situation, in which the nodes are moving disorderly and aimlessly, is best fit for tracking wildlife animal such as ZebraNet \cite{zebranet2004} for tracking zebra, SWIM \cite{Small2003} for tracking whales, Seal-2-Seal \cite{Lindgren2008} to model the social contact patterns of Grey seal or naturally exploiting the animal’s behaviors to develop the feasible routing pattern that not completely random \cite{Yu2007a}.
%% talk about vehicle
In the recent years the vehicle communications have attracted a great deal of attention with an aim to provide connectivity to commuters.
%%
The Vehicle-Infrastructure Connectivity such as \cite{Khabbaz2011,Morris2000,Singh2002,Kun-De2002,Briesemeister2000,Gavrilovich2001,Wang2010,Qi2011,DING201432,Kumar201422} has emerged as a means to enhance traffic safety and reduce the disastrous costs of vehicle collisions \cite{Khabbaz2012} utilizing the opportunistic network concept.
%%
%Talk about Social
The most interesting direction of opportunistic network is the social networking which Exploiting the social behavior of users occupying a large portion of an individual’s daily life to define the basic mechanisms of users’ movements \cite{boldrini2008}.
%
This increasing trend basically comes from the repidly growth of smart mobile devices to enable the concept of \emph{people-centric networking} \cite{Conti2014} since these mobile phones can move, and move with people.
%
Such mobile multihop networks present numerous research challenges such as PeerSoN \cite{Sonja2009,PeerSoN}, PeopleNet \cite{Motani2005}, The Haggle Project \cite{Haggle,Papandrea2009}, COSN framework \cite{Garg2012}.
%
Furthermore, the \emph{opportunistic sensing} such as MetroSense project \cite{Campbell2008} is operated by exploiting the people-centric mobility to sense the devices available in the environment whenever there is a match with the application requirements. 
%
This mobile sensing concept can be used as location-aware data collection instruments for for real world observations \cite{Conti2014}.
%
Moreover, the opportunistic network can be exploited to provide the connectivity in underdeveloped regions.
%
For example, DakNet \cite{Pentland2004} is a project that aims to provide the network connectivity for the remote villages using any connection-enabled vehicle passing by.
%%
In fact, there are several applications in opportunistic networks \cite{Kärkkäinen2013} from different approaches such as mining, message-based applications, stream-based applications or floating content that provides a communication abstraction to allows applications to post contents to a kind of the message boards that offers geographically limited accessibility \cite{Ott2011}.

%=============================================================================
\section{Problem Statement}
\label{intro:Problem Statement}
%=============================================================================


Typically in the OppNets, a Store-Carry-Forward (SCF) technique has been employed on the mobile nodes which enables them to indefinitely carry the messages until they can be further forwarded.
%
In this SCF paradigm, the network suffers the decreasing of performance in the insufficient collaborating nodes environment \cite{Nousiainen2013,Spyropoulos2010}.
%
Since the node holding the data requires next-hop neighbor nodes to forward the data to, the sparse network environment is basically unable to satisfy opportunistic routing.
%
Therefore, such SCF routing technique cannot guarantee 100\% delivery rate. 
%
Moreover, the delivery ratio becomes remarkably low in the sparse network environment especially when there is a strict constraint on message delivery deadline.
%
To the best of our knowledge, the problem of performance degrading in such extreme sparse opportunistic network environments have not been precisely addressed.
%
Considering this problem, there is a need for an innovative protocol design to address this deficiency of OppNets.
%
In this thesis, we propose the algorithms to address the mentioned critical problem of OppNet routing.
%
In the proposed approaches, we use different routing techniques and work on different OppNet scenarios, however our common aim is to increase the delivery ratio in sparse network.

% In addition, none of the traditional routings in OppNets concern about the data content of the messages. 
% If the significance of data is considered as the performance matrix, the network effectiveness of OppNets also drops in sparse network. 
% In several environments, it is essential that the important messages from source to the destination nodes should be specially treated in order to guarantee deliverable.
% Therefore, it is crucial to implement a new protocol to increase the delivery ratio of important data for critical data network such as military tactical network or disaster relief network.


%=============================================================================
\section{Objective and Scope}
\label{intro:Objective and Scope}
%=============================================================================
The objective of this research is to define the methods to increase the delivery ratio of the messages with limited deadline in the opportunistic networks.
%
For this thesis, we focus on the routing protocol implementation with the scope of sparse network environments.

%=============================================================================
\section{Proposed Approaches}
\label{intro:Proposed Approaches}
%=============================================================================
From aforementioned problem statements, this thesis proposed the following approaches:

\begin{itemize}
\item %Dynamic Rendezvous based Routing Algorithm on Sparse Opportunistic Network Environment 
In order to address the problems in the sparse networks, we require to increase the message transferring opportunities as much as possible.
%
In the first approach, we proposed the use of Rendezvous based concept in order to maintain the messages in one place as long as the messages are delivered. 
%
By injecting a special Rendezvous node ($N_{rv}$) into the network, the gap between time and space domain of the mobile nodes can be bridged. 
%
Messages can be transferred from source node to destination node even if they are not in the same location at the same time with the help of the Rendezvous node.
% 
The results clearly show that the delivery ratio of proposed Rendezvous based protocol significant improve over Epidemic protocol especially in the sparse environment.

\item %DORSI: Data-wise Opportunistic Routing with Spatial Information
Nevertheless, the delivery ratio cannot further maximize in the extremely low node density.
%
Therefore, we aim to increase the deliver ability of significant messages to guarantee the critical data deliverable especially the messages with the expiration time constraint. 
%
We proposed a protocol to classify the messages based on the information sensitivity concept along with nodes prioritization technique corresponding to the their delivery probability computed by spatial data. 
%
This protocol classifies the messages according to their significant level, security level and deadline relative to the sensitivity level of data.
% 
In addition we adapts the geographical routing technique to select the best candidate node to forward the messages to the destination. 
%
Simulation experiments clearly illustrate that two key performance indexes: (1) effective delivery ratio and (2) effective replication ratio remarkably improve over the traditional Epidemic routing. 
\end{itemize}

%=============================================================================
\section{Our Contributions}
\label{intro:Our Contributions}
%=============================================================================
In this thesis, we propose two new opportunistic routing algorithms.
%
\begin{itemize}
	\item We propose the Dynamic Rendezvous based Routing Algorithm (DRRA) on Sparse Opportunistic Network Environment which introduces the novel concept of rendezvous place where the passing nodes can announce, deposit or pickup their own messages without having to meet the other nodes carrying the desired message.
	%
	In the proposed scheme, the rendezvous place can be detected automatically and its area’s size and shape are dynamically changed according to the interaction among nodes passing around the area. 
	%
	The results from extensive simulations show that our proposed routing algorithm can achieve higher delivery ratio and utilize lower energy consumption than traditional opportunistic routing algorithms especially in sparse network environment.

	\item We propose the Data-wise Opportunistic Routing with Spatial Information (DORSI), based on the classification level of the data in addition to the spatial information of the nodes. 
	%
	The forwarding algorithm of this routing is determined by significant level, security level and deadline of messages. 
	%
	To conform with actual applications, the scenario of multi-level security in military tactical network is designed as a test bed. 
	%
	Simulation results of DORSI show that the key performances improve over the traditional opportunistic routing. 
	%
	As a result, this novel protocol can guarantee higher delivery ratio on data with higher priority within time limit while restricting the replicas of data with higher security level.

\end{itemize}

%=============================================================================
\section{Thesis Structure}
%=============================================================================

This thesis contains five chapters.
%
Chapter 1 gives an introduction of the research.
%
In addition, the problem statement, objective and scope, and proposed approaches are included in this chapter.
%
In Chapter 2 the background and related works on opportunistic networks are provided.
%
Chapter 3 presents our approach of using the rendezvous place concept to overcome the limitation of insufficient collaborating nodes in sparse network environment.
%
The details of proposed method, simulation model, and performance evaluation are included in this chapter.
%
Chapter 4 describes our message prioritization technique to differentiate the routing based on the significant level of messages.
%
The details of proposed method, simulation model, and performance evaluation are included in this chapter.
%
Chapter 5 includes the discussion, the conclusion and the recommendations for future studies.