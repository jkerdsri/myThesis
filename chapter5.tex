%!TEX root = thesis.tex
\chapter{Conclusions and Discussions}
\label{cf}
% \SVN $Revision: 838 $
% \SVN $Author: sgordon $
% \SVN $Date: 2014-05-12 13:30:42 +0700 (Mon, 12 May 2014) $
% \SVN $URL: https://sandilands.info/svn/Common/Styles/siitthesis/chapter2.tex $


This chapter provides a summary overview of this thesis works and discusses how the proposed algorithms in this thesis can contribute to the field of opportunistic network routing.

%=============================================================================
% \section{Discussion}
% \label{DORSI:Discussion}
%=============================================================================

This thesis has made two major research contributions in opportunistic routing to address the problem in sparse network environments.
%%
First we discuss about how the Rendezvous based routing algorithms can improve the delivery performance of routing.
%%
Then, a data-wise routing in opportunistic networks is discussed.

In chapter \ref{DRRA}, we present the routing technique based on the meeting point of Rendezvous concept to bridge the gap between the space and time domain.
%%
At first, we introduce the \textit{store-carry-forward} paradigm employed by most routing algorithms by which a node can keep the receiving messages, carrying the messages with them when moving and then forwarding the messages copies to the opportunistic meeting nodes when possible.
%%
Then, we point out the problem of most existing routing models since they work well in the networks with high-to-moderate node density in which the opportunity that the moving nodes can meet with each other is rather high.
%%
As a result, most opportunistic routing algorithms perform poorly with delivery ratio becomes remarkably low in the sparse network environment especially when there is strict constraint on message delivery deadline.
%%
Our proposed system introduces the novel concept of rendezvous place where the passing nodes can announce, deposit or pickup their own messages without having to meet the other nodes carrying the desired message.
%%
In the proposed scheme, the rendezvous place can be detected automatically and its area's size and shape are dynamically changed according to the interaction among nodes passing around the area with our proposed \emph{Rumor protocol} and \emph{Sweep protocol}.
%%
The OppNet node can be performed in two operational modes: \emph{Full Power} and \emph{Power Saving} mode, in order to best utilize the power consumption.
%%
For the evaluation, the experiments are performed on two rendezvous place searching algorithms: predictable behavior and non-predictable behavior OppNet nodes.
%%
In this chapter, we also analyzed the delivery performance, power saving factor and rendezvous node factor to the density of OppNet nodes.
%%
By simulation results, we demonstrated that our proposed protocols can improve the delivery performance on the sparse network environment.
%%
This means we can increase the delivery ratio while maintaining the energy utilization.
%%
We believe that the optimum of delivery ratio and delivery performance on the power saving factor can be a significant factor to design an practical applications on the extreme opportunistic networks.

%=============================================================================
% \subsection{Discussion about Chapter \ref{DORSI} on the Data Classification}

Chapter \ref{DORSI} describes a technique to classify the data message in order to differentiate the messages to route differently on the opportunistic networks.
%%
Most of the routing protocols in opportunistic networks consider forwarding decision based solely on locally collected knowledge about node behavior to predict the delivery probability of each node. 
%%
However, only a few of these routing techniques concerns about the data content, and none of them involve the practical scenario of data classification. 
%%
This chapter proposes a novel routing scheme called Data-wise Opportunistic Routing with Spatial Information (DORSI), based on the classification level of the data in addition to the spatial information of the nodes. 
%%
The forwarding algorithm of this routing is determined by significant level, security level and deadline of messages. 
%%
We introduced three key parameters for the routing decision: \emph{priority value}, \emph{replication probability value} and \emph{node ranking value}.
%%
To conform with actual applications in the real world environments, the scenario of multi-level security in military tactical network is designed as a test bed for our simulations. 
%%
In addition, two composite metrics are proposed to analyze the key performance of our designed protocols: \emph{effective delivery ratio} and \emph{effective replication ratio}.
%%
The results show that the key performances improve over the traditional opportunistic routing. 
%%
As a result, this novel protocol can guarantee higher delivery ratio on data with higher priority within time limit while restricting the replicas of data with higher security level.
%%
We believe that this proposed method can further increase the delivery performance of the Rendezvous based routing protocol.
%=============================================================================

All in all, our proposed protocols can address the delivery efficiency issues of sparse opportunistic network environment especially the messages with deadline constraints.
%%
The advancement in the software defined or virtualized nodes has made it possible for our proposed protocols to be implemented in the real world scenarios.












