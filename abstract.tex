% Abstract. 
%
% \SVN $Revision: 840 $
% \SVN $Author: sgordon $
% \SVN $Date: 2014-05-12 13:39:14 +0700 (Mon, 12 May 2014) $
% \SVN $URL: https://sandilands.info/svn/Common/Styles/siitthesis/abstract.tex $

Opportunistic Network (OppNet) is a challenge network exploiting contact opportunities and node mobility to route the messages even a complete path from source to destination never exists.
%%
The example applications for such extreme networks are in the environments of battlefield network, wildlife monitoring or disaster response  where movements are random with highly intermittent connections. 
%5
This opportunistic routing relies on store-carry-forward paradigm, which a data holding node can carry the data and find an opportunity to forward data while moving via encountering nodes until the data reaches to the destination.
%5
However, the performance of such store-carry-forward scheme largely depends on node encountering opportunity which will become lower in more sparse network environment.
%
There are several proposed routing algorithms in the literature but only few have addressed routing problems in high sparse network environments especially with strict constraints in energy consumptions and message delivery deadlines.
%
In order to improve the delivery ratio in such sparse network while minimizing the energy consumption, we proposed a novel Dynamic Rendezvous based Routing Algorithm on Sparse Opportunistic Network Environment where the rendezvous concept is implemented to increase indirect node encountering opportunity in such extreme environment.
%5
In addition, we proposed DORSI: Data-wise Opportunistic Routing with Spatial Information where more significant data has more chances to be forwarded and occupies system resources.
%
This algorithm leads to higher effective delivery ratio in the system.
%
The extensive simulations are used to evaluate the performance of the proposed algorithms compared to the tradition OppNet routing algorithms.


