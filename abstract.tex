% Abstract. 
%
% \SVN $Revision: 840 $
% \SVN $Author: sgordon $
% \SVN $Date: 2014-05-12 13:39:14 +0700 (Mon, 12 May 2014) $
% \SVN $URL: https://sandilands.info/svn/Common/Styles/siitthesis/abstract.tex $

Opportunistic Network (OppNet) is a challenge network exploiting contact opportunities and node mobility to route the messages even a complete path from source to destination never exists.
The example applications for such extreme networks are in the environments of battlefield network, wildlife monitoring or disaster response  where movements are random with highly intermittent connections. 
This opportunistic routing relies on store-carry-forward paradigm which a data holing node such as source or neighbor node can carry the data and finds an opportunity to forward data by discovering its nearest neighbor node and uses it to forward messages toward the destination node.
However, the performance of opportunistic routing algorithm largely depends on several factors such as limited knowledge of contact behavior or the density of mobile nodes.
The problems arise in sparse network environment with limited delivery deadline results in low delivery ratio.
Several researches attempted to address the sparseness problem by a special node such as data mules or message ferries.
Nevertheless, proposed solutions impractical under some application environments especially with limited power constraints.

In order to improve the delivery ratio in such sparse network while maintaining the energy consumption, we proposed a novel Dynamic Rendezvous based Routing Algorithm on Sparse Opportunistic Network Environment where the rendezvous concept is implemented to address the problem of routing in sparse environment.
In addition, we proposed DORSI: Data-wise Opportunistic Routing with Spatial Information where the significant of data content is accounted for the forwarding algorithm of the nodes. 
This DORSI can improve the delivery ratio for the important messages thus increase the delivery ratio if the weight of each class is accounted.
In those algorithms, our common objective is to increase the network performance such as delivery ratio or composite matrices under given circumstances.
We also present intensive simulation results regarding the performance comparison of the proposed algorithms with the tradition OppNet routing algorithms.


